We use C\+Make which is a cross-\/platform solftware managing the build process. First, create a build directory in R\+O\+H\+S\+A-\/\+G\+PU and go inside this directory.

mkdir build \&\& cd build

Using C\+Make we can produce a Makefile. We may run it using make.

cmake ../ \&\& make

When launching the program, we need to specify the parameters.\+txt file filled in by the user \+: ~\newline


./\+R\+O\+H\+S\+A-\/\+G\+PU parameters.\+txts

\mbox{\hyperlink{main_8cpp_source}{main.\+cpp}} code

Reading the user file parameters.\+txt

We declare an parameters-\/type object that will the parameters.\+txt file.

parameters user\+\_\+parametres(argv\mbox{[}1\mbox{]});

Getting the hypercube.

We can read the F\+I\+TS or D\+AT file by using the class hypercube.

hypercube Hypercube\+\_\+file(user\+\_\+parametres, user\+\_\+parametres.\+slice\+\_\+index\+\_\+min, user\+\_\+parametres.\+slice\+\_\+index\+\_\+max, whole\+\_\+data\+\_\+in\+\_\+cube);

Getting the gaussian parameters (call to the R\+O\+H\+SA algorithm).

The class \mbox{\hyperlink{classalgo__rohsa}{algo\+\_\+rohsa}} runs the R\+O\+H\+SA algorithm based on the two objects previously declared.

\mbox{\hyperlink{classalgo__rohsa}{algo\+\_\+rohsa}} algo(user\+\_\+parametres, Hypercube\+\_\+file);

Plotting and storing the results.

We can plot the smooth gaussian parameters maps and store the results back into a F\+I\+TS file by using some of the routines of the class hypercube.

\mbox{\hyperlink{classalgo__rohsa}{algo\+\_\+rohsa}} algo(user\+\_\+parametres, Hypercube\+\_\+file); 